\section{Build an \cUbuntu machine on HPC}

\bxsh{Login to NREC HPC facility}{ssh -i \hm/.ssh/id_rsa ubuntu@158.37.63.141}

\subsection{Install \cR and associated packages}

\bxsh{Let \cUbuntu know from which list to download the latest \cR (using \CVim)}{sudo vim /etc/apt/sources.list}

\bxvim{Add \cR repo}{
    deb https://cloud.r-project.org/bin/linux/ubuntu focal-cran40/\\
    deb http://ftp.uninett.no/ubuntu/ bionic-backports main restricted universe
}

\begin{caution}{Attention}
    The second line starts with ``http'', \emph{not} ``https''.
\end{caution}

\bxsh{Let \cUbuntu know this key is authentic so don't freak out about security}{
    sudo apt-key adv --keyserver keyserver.ubuntu.com --recv-keys E298A3A825C0D65DFD57CBB651716619E084DAB9
}

\bxsh{Install \CR}{
    sudo apt-get update\\
    sudo apt-get install r-base
}

\bxsh{Install some \cLinux libraries in order to enable \cR to build its own packages}{
    sudo apt-get install libgsl-dev libgit2-dev libssl-dev libcurl4-openssl-dev libxml2-dev libudunits2-dev libfontconfig1-dev libcairo2-dev libgeos-dev libgdal-dev libgmp_dev libharfbuzz-dev libfribidi-dev
}

\bxsh{Get into \CR}{R}

\bxr{Install the first package}{
    install.packages("data.table", dependencies = TRUE)
}

\begin{note}{\CR's personal library}
Answer yes to both questions so that a personal library can be created and installation process can begin.
\end{note}

\bxsh{Install \CGit}{
    sudo apt install git\\
    git config --global user.name "Tony Tan"\\
    git config --global user.email "tctan@uio.no"
}

\bxsh{Manually install an \cR package \pk{colorout} because it is not in \pk{CRAN}}{
    git clone https://github.com/jalvesaq/colorout.git\\
    R CMD INSTALL colorout
}

\bxsh{Tell \cR to automatically load \pk{colorout} each time it starts}{
    cd\\
    touch .Rprofile
}

\bxsh{Open \pk{.Rprofile} using \CVim}{
    vim .Rprofile
}

\bxvim{Autoload \cR packages}{
    old <- getOption("defaultPackages")\\
    options(defaultPackages = c(old, "colorout"))
}

\bxsh{Install Java Developer Kit}{
    sudo apt-get install default-jre default-jdk\\
    sudo R CMD javareconf
}

\bxsh{Go back to \cR and continue installing packages}{R}

\begin{note}{You may be interested in the following \cR packages}
\begin{itemize}
    \item \pk{BiocManager}: enable package installation from Bioconductor
    \item \pk{remotes}: enable package installation from github
    \item \pk{Orcs}: manage working directories for multiple operating systems
    \item \pk{MplusAutomation}: link \cR with \pk{Mplus}
    \item \pk{intsvy}: for analysing ILSA data
    \item \pk{mice}, \pk{micemd}, \pk{miceadds}, \pk{jomo}: for multiple imputation
    \item \pk{ggplot2}, \pk{tidyr} etc.: commonly used \cR packages
    \item \pk{rhdf5}: \pk{MplusAutomation} needs this to function. N.B. This package must be installed from \pk{Bioconductor}
\end{itemize}
\end{note}

\bxr{Continue package installation}{
    install.packages(c("BiocManager", "remotes", "Orcs", "MplusAutomation", "intsvy", "mice", "micemd", "miceadds", "jomo", "ggplot2", "tidyr", "dplyr", "tibble", "stringr", "shiny", "shinythemes", "modelr", "mlr"; "readr"), dependencies = TRUE)\\
    BiocManager::install("rhdf5")
}

\begin{note}{\cR colouration}
Now you should see \cR output in different colours.
\end{note}


\bxr{Update all packages, then quit \CR}{
    update.packages(ask = FALSE, checkBuilt = TRUE)\\
    q()\\
    n
}

\subsection{Install necessary \cPy packages}

\subsubsection{Install \pk{pip}}

\bxsh{Check which version of \cPy HPC already has}{
    python3 --version
}

\begin{note}{\cPy version}
    You should see a one-line output ``Python 3.8.5''
\end{note}

\bxsh{Install \pk{pip}}{
    sudo apt install python3-pip
}

\subsubsection{Install \pk{radian}}

\bxsh{Install \pk{radian}}{
    pip3 install -U radian
}

\begin{note}{Add \pk{radian} to path}
    We need to tell \cLinux where \pk{radian} lives so that it can be triggered automatically each time we type ``r'' shorcut. We need to modify two files: \pk{.profile} and \pk{.bashrc}.
\end{note}

\bxsh{Get into \pk{.profile}}{
    vim .profile
}

\bxvim{Add \pk{radian} path to \pk{.profile}}{
    \# Use radian to better manage R interface\\
    export PATH=\$PATH:/home/ubuntu/.local/bin/
}

\bxsh{Get into \pk{.bashrc}}{
    vim .bashrc
}

\bxvim{Create some shortcuts}{

    \# Use r key as shortcut to radian\\
    alias r="radian"\\

    \# Update Python3\\
    alias pyupdate="sudo pip3 list --outdated --format=freeze | grep -v '\textasciicircum\textbackslash -e' | cut -d = -f 1 | xargs -n1 pip3 install -U"\\

    \# Create shortcut to UiO M Drive\\
    alias m="sshfs tctan@login.uio.no: ~/uio -o reconnect,modules=iconv,from_code=ISO-8859-1,ConnectTimeout=10"
}

\bxsh{Create a new file \pk{.bash_profile}}{
    touch .bash_profile
}

\bxvim{Bring \pk{.profile} and \pk{.bashrc} together}{
    . \hm/.profile\\
    . \hm/.bashrc
}

\bxsh{Refresh these files}{
    source .bash_profile\\
    source .profile\\
    source .bashrc
}

\subsubsection{Update \CPy}

\bxsh{Install some \cLinux libraries so that \cPy will not complain}{
    sudo apt install python3-testresources libcups2-dev libgirepository1.0-dev libparted-dev
}

\bxsh{Update \CPy}{
    pyupdate
}

\subsection{Link UiO M Drive to HPC}

\bxsh{Create a receiving folder named ``uio'' in the home directory}{
    mkdir uio
}

\bxsh{Install \pk{sshfs}}{
    sudo apt install sshfs
}

\bxsh{Login to M Drive using shortcut ``m''}{
    m
}

\begin{note}{ECDSA key fingerprint}
    Type ``yes'' to the warning:

    The authenticity of host 'login.uio.no (129.240.12.7)' can't be established.

    ECDSA key fingerprint is SHA256:efy2GuDcvOC2HEe0PwkuCaYVL8tKC77Bi8uHM+m4K6s.

    Are you sure you want to continue connecting (yes/no/[fingerprint])?

    \bigskip

    If you receive ``read: Connection reset by peer'', type shortcut ``m'' again to re-connect---it only happens the very first time M Drive is established.
\end{note}

\subsection{Install \cM}

\begin{caution}{Switch to local}
    Switch back to local \cLinux home directory for the following steps. (If you are in HPC environment, type ``exit'' to switch back to local machine.)
\end{caution}

\bxsh{Get into \pk{.bashrc}}{
    vim .bashrc
}

\bxvim{Link local \cLinux to HPC}{
    \# Logon to HPC\\
    alias hpc="ssh -i \hm/.ssh/id_rsa ubuntu@158.37.63.141"
}

\bxsh{Renew \pk{.bashrc} and push some local files to HPC}{
    source .bashrc
}

\begin{caution}{Switch to HPC}
    Now switch back to HPC terminal. (In local environment, type ``hpc'' to switch to HPC.)
\end{caution}

\bxsh{Copy \cM installer from M Drive to home directory}{
    m\\
    (Type your UiO M Drive password)\\
    cp /home/tony/uio/pc/Dokumenter/ComboLinux64.bin /home/tony/
}

\bxsh{Set up \cM installation folder}{
    cd /opt\\
    sudo mkdir mplus\\
    sudo chmod 777 mplus\\
    cd mplus\\
    sudo mkdir 8.5\\
    sudo chmod 777 8.5\\
    cd
}

\bxsh{Install \cM}{
    chmod a+x ComboLinux64.bin\\
    ./ComboLinux64.bin
}

\begin{note}{Accept all default settings}
    Follow instructions on the screen. Press ENTER when asked.
\end{note}

\subsection{Update and reboot HPC}

\bxsh{Update and reboot}{
    sudo apt-get update\\
    sudo apt-get upgrade\\
    sudo reboot
}

\begin{note}{Let HPC reboot}
    Give a couple of minutes for HPC to reboot. Then your HPC would be ready.
\end{note}

\subsection{Interact with HPC from local \cLinux computer}

\bxsh{Logon to HPC via shortcut}{
    hpc
}

\bxsh{Monitor HPC performance}{
    hpc\\
    htop
}

